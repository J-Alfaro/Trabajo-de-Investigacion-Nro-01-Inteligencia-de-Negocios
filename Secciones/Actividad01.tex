%%%%%%%%%%%%%%%%%%%%%%%%%%%%%%%%%%%%%%%%%%%%%%%%%%%%%%%%%%%%%%%%%%%%%%%%%%%
%
% Plantilla para un artículo en LaTeX en español.
%
%%%%%%%%%%%%%%%%%%%%%%%%%%%%%%%%%%%%%%%%%%%%%%%%%%%%%%%%%%%%%%%%%%%%%%%%%%%



%--------------------------------------------------------------------------
\title{Plantilla para un artículo \LaTeX}
\author{El autor va aquí\\
  \small Dept. Plantillas y Editores\\
  \small E12345\\
  \small España
}

\begin{document}
\maketitle

\abstract{Esto es una plantilla simple para un artículo en \LaTeX.}
\\
\abstract{Esto es una plantilla simple para un artículo en \LaTeX.}

\section{Introduccion}

Aquí va el texto.
\begin{equation}\label{eq:area}
  S = \pi r^2
\end{equation}
Uno puede referirse a ecuaciones así: ver ecuación (\ref{eq:area}).
También se pueden mencionar secciones de la misma forma: ver sección
\ref{sec:nada}. O citar algo de la bibliografía: \cite{Cd94}.

\section{Marco teorico}

Aquí va el texto.
\begin{equation}\label{eq:area}
  S = \pi r^2
\end{equation}
Uno puede referirse a ecuaciones así: ver ecuación (\ref{eq:area}).
También se pueden mencionar secciones de la misma forma: ver sección
\ref{sec:nada}. O citar algo de la bibliografía: \cite{Cd94}.

\section{Aplicacion}

Aquí va el texto.
\begin{equation}\label{eq:area}
  S = \pi r^2
\end{equation}
Uno puede referirse a ecuaciones así: ver ecuación (\ref{eq:area}).
También se pueden mencionar secciones de la misma forma: ver sección
\ref{sec:nada}. O citar algo de la bibliografía: \cite{Cd94}.




\subsection{Subsection}\label{sec:nada}

Más texto.

\subsubsection{Subsubsection}\label{sec:nada2}

Más texto.


\section{Conclusiones}

(\ref{eq:area}).
También se pueden mencionar secciones de la misma forma: ver sección
\ref{sec:nada}. O citar algo de la bibliografía: \cite{Cd94}.


% Bibliografía.
%-----------------------------------------------------------------
\begin{thebibliography}{99}

\bibitem{Cd94} Autor, \emph{Título}, Revista/Editor, (año)

\end{thebibliography}

\end{document}
